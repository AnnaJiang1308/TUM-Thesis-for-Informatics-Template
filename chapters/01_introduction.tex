% !TeX root = ../main.tex

\chapter{Introduction}\label{chapter:introduction}
Although High Driving Automation(SAE Level 4) is achievable in the foreseeable future\cite{inagaki2019critique}, most drivers nowadays still prefer comprehensive control over their own vehicles. Therefore, increasingly more studies have been focusing on driving safety\cite{lee2005driving,lee2008fifty}. 
These papers show that driver behavior is the primary cause of driving accidents. In that case, researchers have developed several methods to avoid potential danger. Such methods involve either improving monitoring of the vehicle's inside situation, which analyzes the driving parameters as well as the driver him or herself to determine whether there is no abnormality\cite{karrouchi2023driving} or proposing a vehicle detection and tracking system from an outer view that estimates time-to-collision (TTC) and warns the driver for a possible collision\cite{aytekin2010increasing}.

On the other hand, little research focuses on driving behavior prediction, which may be due to the lack of application of the undetermined decision model in behavior prediction, specifically under autonomous driving. Scientists have succeeded in years of generating dynamic graphs based on videos. However, there is still a gap in the rational use of these forms for behavioural prediction.

This work would like to focus on constructing a comprehensive behaviour prediction model based on graph neural networks (GNNs). A Graph Neural Network is a novel type of neural network architecture that can be applied to graph-like inputs. As the target is to train the behavior model for the participators, the training data would contain interaction between several objects and participators while driving. GNN is, therefore, prioritized due to its unique structure. The work would take special care in building temporal dynamic graphs, as our model will be trained based on sequences of behavior descriptions extracted from videos.

In addition, to ensure the compatibility of the dataset and training model,  some adaptations are made based on JODIE\cite{kumar2019predicting}, which is the model based on the dynamic evolution of users and items. To make the predictions as detailed as possible, I expanded the output catalog to ensure that more detailed information about the predicted behavior would be represented in the output.

In a nutshell, I would first gather all the necessary information with the help of the Large Language model (LLM), convert it into dynamic graphs, and insert them into the model I have adapted from model JODIE. These graphs could contain either one specific participant or all the people involved. By learning how dynamic graphs change under time series, like the appearance and vanish of all these edges and nodes in the graph, the model should be able to absorb the feature behind it and come up with the prediction for behaviour in the future. Therefore, anomaly detection could also be achieved by comparing the predicted graph with the real one and alerting once when unsuitable behaviour during driving is detected. This model provides a new perspective for predicting driving behaviour detected in videos and allows for more diversified and targeted forecasting.


\section{contribution}
The main contributions of this thesis are summarized as follows:
\begin{description}
    \item[Dataset Collecton] From the Dataset \textit{drive \& act}, I acquire the hierarchical activity labels of given video data and rewrite them as time sequences. I also cluster the behaviour types into several categories with the help of the Large Language Model.
    \item[Dynamic Graph Construction] By reassembling the nodes and edges in the video data, I construct time-sequenced dynamic graphs that could be used as training data for the model. Such a graph captures complex dependencies and offers a hierarchical representation abstracted from the raw data.
    \item[Learning Model Adaption] To enrich the diversity of the prediction, I expand the output of the dynamic graph-based learning model from binary to catalogue description to ensure that not only the interaction itself but also the type of it would be described.
    \item[anormaly detection] By comparing the predicted model with the real one, I could detect any unsuitable behaviour during the driving and alert the driver. Differ from the research before, our 
\end{description}
    
    



\section{Structure}

This thesis is structured as follows. In Chapter\ref{chapter:background}, I would introduce and explain concepts and definitions concerning this document. Chapter\ref{chapter:relatedwork} reviews related work in anomaly detection and dynamic link prediction models, the dataset this work refers to, and the model I have adapted. Chapter\ref{chapter:methodology} describes the methodology of this work, including the dataset collection, dynamic graph construction, and model adaption. Chapter\ref{chapter:evaluation} presents the evaluation of the model and the results of the prediction. Chapter\ref{chapter:futurework} discusses the potential future work that could be done based on this work. Chapter\ref{chapter:conclusion} concludes the thesis and summarises the work done.





