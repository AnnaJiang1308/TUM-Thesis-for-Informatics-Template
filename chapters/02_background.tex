

\chapter{Background}\label{chapter:background}

\section{Scene Graph}
The scene graph is a structured representation of a scene that can clearly express the objects, attributes, and relationships between objects in the scene\cite{9661322}. Accompanied by the development of computer vision technology, not simply detecting and recognizing objects in images no longer satisfies the researchers, as they would expect some higher level of understanding and reasoning for image vision tasks. In this way, an intuitive idea comes up about adding up the relationship between the detected objects. The earliest research could dated back to 2017, when some objects and relations of a given image could be inferred and a scene graph would be produced as a result\cite{xu2017scene}. Later in 2018,


\section{Large Language Model}
Large Language Models (LLMs) are highly complex artificial intelligence systems that can learn from the vast amounts of available text data\cite{radford2018improving}. Thanks to the attending of \textit{Transformer} \cite{vaswani2017attention}, a deep learning architecture, these language models which employed self-supervised pre-training have demonstrated improved efficiency and scalability in many fields.
 Based on self-attention mechanisms and feed-forward module,\textit{Transformer} has overwhelming advantages in computing representations and global dependencies.

In concrete terms, the Large Language Models equipped with \textit{Transformer} are capable of diverse tasks raised by Natural Language Processing\cite{chowdhary2020natural}, such as textual entailment, question answering, semantic similarity assessment, and document classification. Take BERT\cite{alaparthi2020bidirectional} and GPT \cite{radford2018improving,radford2019language,brown2020language} as two examples, the former utilizes transformer encoder blocks to predict missing words in a given text, and the latter has been enjoying a tremendous reputation for generating diverse and human-like responses, showcasing its potential in various domains.

    \subsection{Text Classification}
In our project, a bunch of hierarchical activity labels would be acquired from the dataset \textit{drive \& act} to depict every detail of the participant's movements in the driving behavior recorded in the video. These labels, however, are too trivial for the construction of learning data, as considering each activity individually will be tedious in such a vast and complex model training process. Therefore, labels should be classified according to the object on which this behavior operates or the specificity of the moment in which the action takes place. For example, the fastening of a seatbelt should occur shortly after entering the vehicle. To accomplish such a task, 


\section{graph}




    \subsection{Dynamic graph}

    \subsection{Graph Neural Networks}

\section{behavior prediction}

