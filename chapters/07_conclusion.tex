\chapter{Conclusion}\label{chapter:conclusion}

In this work, we proposed a novel approach to predicting driver behavior based on dynamic graph neural networks. Using the \textit{Drive\&Act} dataset, we reorganized detected data from source videos into time-sequenced dynamic graphs. After extensive comparisons, JODIE was selected from a range of dynamic graph learning models due to both theoretical and experimental evidence and was applied in our work. To enable more detailed predictions, we adapted JODIE by introducing edge state embeddings to predict driver behavior more accurately. Additionally, the model’s output was expanded to provide categorical descriptions, ensuring that both interactions and their types are captured. Finally, this adapted model was compared against traditional methods, such as HMM, to assess its performance in detecting abnormal behaviors that could signal potential accidents.

The evaluation results reveal several key insights. Firstly, consistent with earlier findings, JODIE performs exceptionally well compared to other models across multiple metrics. Secondly, although incorporating state embeddings led to a slight drop in AP and AUROC scores, this trade-off is justified as the model gains the ability to incorporate richer features and provide more detailed predictions. Lastly, our adapted model demonstrates clear advantages over the HMM in most cases, providing strong evidence of its superiority over traditional approaches.

Looking ahead, there are several areas for improvement. Evidence suggests that, when paired with the \textit{Drive\&Act} dataset, the model shows signs of underfitting. This highlights the need for more extensive datasets, including longer time sequences and more detailed graph structures with hierarchical attributes. Furthermore, the integration of the model with hardware systems poses additional challenges. Issues such as computational capacity and real-time compatibility need to be addressed to ensure seamless deployment.

Overall, the model proposed in this work demonstrates that dynamic graph neural networks are a promising solution for driver behavior prediction. This approach holds significant potential for improving safety and advancing autonomous driving systems.

