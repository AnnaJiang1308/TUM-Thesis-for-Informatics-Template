\chapter{\abstractname}

Despite significant advances in autonomous driving, improving driver safety remains a critical focus. Research has been done to detect potential risks by monitoring driver behavior or the surrounding environment. However, no specific method has incorporated logistic sequential analysis for behavior predictions. In my master thesis, I propose a novel approach using dynamic graphs to predict anomaly behavior in the driving process. I selected JODIE, a model well suited to capturing driver-object interactions due to its compatibility with our dataset structure. I adapted JODIE by incorporating state attributes into graph edges to enhance prediction capabilities. The modified model predicts the edges that would appear or vanish and their associated attributes in the next time sequence. I collected and transformed video data of driver participants into dynamic graphs, representing interactions between drivers and objects over time. These dynamic graphs were then used to train and evaluate the adapted model. Our approach achieved promising performance in detecting anomalies in driver behavior. The results include predictions of objects the interactions would contain and the corresponding behavior types. Such outcomes would be more detailed and provide valuable insights into potential risks. This work presents a potential solution for advancing autonomous driving systems and contributes to a safer driving environment.



\makeatletter
\ifthenelse{\pdf@strcmp{\languagename}{english}=0}
{\renewcommand{\abstractname}{Kurzfassung}}
{\renewcommand{\abstractname}{Abstract}}
\makeatother

\chapter{\abstractname}

%TODO: Abstract in other language
\begin{otherlanguage}{ngerman} % TODO: select other language, either ngerman or english !
    Trotz erheblicher Fortschritte beim autonomen Fahren bleibt die Verbesserung der Fahrsicherheit ein wichtiger Schwerpunkt. Forschungen haben gezeigt, dass es möglich ist, Risiken durch die Überwachung von Elementen wie dem Fahrerverhalten und der Umgebung zu erkennen. Allerdings gibt es keine spezifische Methode, die eine logistische sequentielle Analyse für zukünftige Vorhersagen beinhaltet. In dieser Arbeit schlage ich einen neuen Ansatz zur Erkennung von Anomalien im Fahrerverhalten unter Verwendung dynamischer Graphen vor. Ich wählte JODIE, ein Modell, das sich aufgrund seiner Kompatibilität mit der Struktur unseres Datensatzes gut zur Erfassung von Fahrer-Objekt-Interaktionen eignet. Um die Vorhersagefähigkeiten zu verbessern, habe ich JODIE angepasst, indem ich Zustandsattribute in Graphenkanten integriert habe. Das modifizierte Modell sagt nicht nur die Kanten voraus, die in der nächsten Zeitsequenz auftreten würden, sondern auch die zugehörigen Attribute. Ich sammelte und organisierte Videodaten von Fahrern zu dynamischen Graphen, die die Interaktionen zwischen Fahrern und Objekten über die Zeit darstellen. Diese dynamischen Graphen wurden dann zum Trainieren und Bewerten des angepassten Modells verwendet. Unser Ansatz erzielte eine vielversprechende Leistung bei der Erkennung von Anomalien im Fahrerverhalten. Die Ergebnisse umfassen Vorhersagen zu den Objekten, die in den Interaktionen enthalten sind, und zu den entsprechenden Verhaltenstypen. Solche Ergebnisse wären detaillierter und würden wertvolle Erkenntnisse über potenzielle Risiken liefern. Diese Arbeit stellt eine potenzielle Lösung für die Weiterentwicklung autonomer Fahrsysteme dar und trägt zu einer sichereren Fahrumgebung bei.

\end{otherlanguage}


% Undo the name switch
\makeatletter
\ifthenelse{\pdf@strcmp{\languagename}{english}=0}
{\renewcommand{\abstractname}{Abstract}}
{\renewcommand{\abstractname}{Kurzfassung}}
\makeatother