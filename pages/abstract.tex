\chapter{\abstractname}

Despite significant advancements in autonomous driving, improving driving safety remains a critical focus. Researches have demonstrated the potential to detect risks by monitoring elements such as driver behavior and the surrounding environment. However, no specific method has incorporated logistic sequential analysis for future predictions. In this thesis, I propose a novel approach to anomaly detection in driver behavior using dynamic graphs. I selected JODIE, a model well-suited to capturing driver-object interactions due to its compatibility with our dataset structure. To enhance prediction capabilities, I adapted JODIE by incorporating state attributes into graph edges. The modified model predicts not only the edges that would appear but also their associated attributes in the next time sequence. I collected and organized video data of driver participants into dynamic graphs, which represent interactions between drivers and objects over time. These dynamic graphs were then used to train and evaluate the adapted model. Our approach achieved promising performance in detecting anomalies in driver behavior. The results include predictions of object the interactions would contain and the corresponding behavior types. Such outcomes would be more detailed and provide valuable insights into potential risks. This work presents a potential solution for advancing autonomous driving systems, contributing to safer driving environments.



\makeatletter
\ifthenelse{\pdf@strcmp{\languagename}{english}=0}
{\renewcommand{\abstractname}{Kurzfassung}}
{\renewcommand{\abstractname}{Abstract}}
\makeatother

\chapter{\abstractname}

%TODO: Abstract in other language
\begin{otherlanguage}{ngerman} % TODO: select other language, either ngerman or english !
    Trotz erheblicher Fortschritte beim autonomen Fahren bleibt die Verbesserung der Fahrsicherheit ein wichtiger Schwerpunkt. Es wurde gezeigt, dass Methoden zur Erkennung von Risiken durch die Überwachung von Elementen wie dem Fahrerverhalten und der Umgebung möglich sind. Allerdings hat keine spezifische Forschung die logistische sequentielle Analyse für zukünftige Vorhersagen einbezogen. In dieser Arbeit schlagen wir einen neuartigen Ansatz zur Erkennung von Anomalien im Fahrerverhalten unter Verwendung dynamischer Graphen vor, der eine neue Perspektive auf dieses Thema bietet. Nach der Evaluierung modernster dynamischer Graphen-Lernmodelle haben wir uns für JODIE entschieden, ein Modell, das sich aufgrund seiner Kompatibilität mit der Struktur unseres Datensatzes gut für die Erfassung von Fahrer-Objekt-Interaktionen eignet. Um die Vorhersagefähigkeiten zu verbessern, haben wir JODIE angepasst, indem wir Zustandsattribute in die Graphenkanten aufgenommen haben. Das modifizierte Modell sagt nicht nur die Kanten voraus, die in der nächsten Zeitsequenz auftreten werden, sondern auch die zugehörigen Attribute. Hier haben wir Videodaten von Fahrern gesammelt und in dynamische Graphen organisiert, die die Interaktionen zwischen Fahrern und Objekten über die Zeit darstellen. Diese dynamischen Graphen wurden dann zum Trainieren und Bewerten des angepassten Modells verwendet. Unser Ansatz erzielte eine vielversprechende Leistung bei der Erkennung von Anomalien im Fahrerverhalten. Die Ergebnisse umfassen Vorhersagen zu den Objekten, die in den Interaktionen enthalten sind, und zu den entsprechenden Verhaltenstypen. Solche Ergebnisse wären detaillierter und würden wertvolle Erkenntnisse über potenzielle Risiken liefern. Diese Arbeit stellt eine potenzielle Lösung für die Weiterentwicklung autonomer Fahrsysteme dar und trägt zu einer sichereren Fahrumgebung bei.

\end{otherlanguage}


% Undo the name switch
\makeatletter
\ifthenelse{\pdf@strcmp{\languagename}{english}=0}
{\renewcommand{\abstractname}{Abstract}}
{\renewcommand{\abstractname}{Kurzfassung}}
\makeatother